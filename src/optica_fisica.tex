\section{Óptica Física}
\capequation{Interferencia - Experiencia de Young}
\begin{itemize}
    \item Fenómeno de interferencia absoluta
    \item En una pantalla lejana (frente de ondas planos) a las rendijas (las cuales a su vez se comportan como fuentes por el principio de Huygens) se observa un patrón de interferencia provocado por una o más fuentes \textbf{coherentes}.
    \item \textbf{Fuentes coherente}: Fuente monocromática, diferencia de fase constante
    \item Máximos
    \begin{equation}
        x^{max} = n \cdot \frac{\lambda D}{d} , n \in \mathbb Z
    \end{equation}
    \item Mínimos (N: Número de fuentes)
    \begin{equation}
        x^{min} = \frac{n}{N} \cdot \frac{\lambda D}{d} , n \in \mathbb Z \quad \land \quad n \neq N
    \end{equation}
    \item Otra forma de calcular los máximos y mínimos en función del ángulo $\theta$x
    \begin{equation}
    \begin{split}
       d \sin(\theta^{max}) &= n \lambda \\
       d \sin(\theta^{min}) &= \frac{2n + 1}{2} \lambda 
    \end{split}
    \end{equation}
    \item Cantidad de máximos principales y secundarios
    \begin{itemize}
        \item Principales:
        \begin{equation}
            N^o_{ppal} = N - 1
        \end{equation}
        \item Secundarios:
        \begin{equation}
            N^o_{sec} = N - 2
        \end{equation}
    \end{itemize}
    \item Intensidad Máxima (ángulos pequeños) (N: Número de fuentes)
    \begin{equation}
        I = N^2 \cdot I_0
    \end{equation}
\end{itemize}

\capequation{Difracción}
\begin{itemize}
    \item Mínimos de difracción (\textbf{b}: ancho de rendija. Similar al parámetro d en interferencia)
    \begin{equation}
        x^{min} = n \cdot \frac{\lambda D}{b} , n \in \mathbb Z^*
    \end{equation}
    \item Mínimos de difracción en función del ángulo
    \begin{equation}
        b \sin(\theta^{min}) = n \lambda , n \in \mathbb Z^*
    \end{equation}
\end{itemize}

\capequation{Redes de difracción}
\begin{itemize}
    \item Constante de la red (\textbf{d}: separación de ranuras. No confundir con ancho de ranuras \textbf{b})
    \begin{equation}
        C = \frac{1}{d}
    \end{equation}
    \item Máximo orden observable
    \begin{equation}
        n_{max} = \frac{d}{\lambda}
    \end{equation}
\end{itemize}