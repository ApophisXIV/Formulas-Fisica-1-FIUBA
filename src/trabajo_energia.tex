
\section{Trabajo y energía}
    \capequation{Trabajo de una fuerza}
    \begin{equation}
        W_F = \int_{r_1}^{r_2} \Vec{F} \boldsymbol{\cdot} d\vec{r}
    \end{equation}
    Para una trayectoria recta...
    \begin{equation*}
        W_F = F \cdot \Delta x
    \end{equation*}

    \newpage
    
    \capequation{Potencia Media}
    \begin{equation}
        P_m = \frac{W_F}{\Delta t}
    \end{equation}
    
    \capequation{Potencia Instantánea}
    \begin{equation}
        P_i = \Vec{F} \boldsymbol{\cdot} \vec{v}_{inst}
    \end{equation}
    
    \capequation{Energía cinética (m=cte)}
    \begin{equation}
        Ec = \frac{1}{2}\cdot m \cdot v^2
    \end{equation}
    \begin{equation*}
        \Delta Ec = \frac{1}{2}\cdot m \cdot ({v_f}^2 - {v_0}^2)
    \end{equation*}
        
    \capequation{Energía potencial gravitatoria (m = cte)}
    \begin{equation}
        Epg = m \cdot g \cdot h
    \end{equation}
    \begin{equation*}
        \Delta Epg = m \cdot g \cdot (h_f - h_0)
    \end{equation*}

    \capequation{Energía potencial elástica}
    \begin{equation}
        Epe = \frac{1}{2} \cdot k \cdot x^2
    \end{equation}
    \begin{equation*}
        \Delta Epe = \frac{1}{2} \cdot k \cdot (x_f^2 - x_0^2)
    \end{equation*}
    \capequation{Teorema de las fuerzas vivas (m = cte)}
    \begin{equation}
        W_{TF} = \Delta Ec
    \end{equation}
    
    \capequation{Teorema de las fuerzas conservativas (m = cte)}
    \begin{itemize}
        \item Campos de fuerza conservativos. Una forma de intuir si un campo de fuerzas es conservativo es mediante el teorema de Schwarz. 
        \item Si es 2D... $\left( \frac{\partial f_y}{\partial x} = \frac{\partial f_x}{\partial y} \right)$
    \end{itemize}
    \begin{equation}
        W_{FC} = -\Delta Ep
    \end{equation}

    \capequation{Teorema de las fuerzas no conservativas}
    \begin{equation}
        W_{FNC} = \Delta Em
    \end{equation}
    \begin{equation*}
        \Delta Em = \Delta Ec +\Delta Ep
    \end{equation*}
    
    \capequation{EP en un campo de fuerza conservativo}
    \begin{equation}
        \vec{F} = -\left(\frac{\partial F}{\partial x} , \frac{\partial F}{\partial y} , \frac{\partial F}{\partial z} \right) = -\nabla Ep
    \end{equation}

    \newpage
    % \capequation{Diagramas energéticos}
    % \begin{equation}
    %     \begin{split}
    %         \\
    %         \\
    %         \\
    %     \end{split}
    % \end{equation}
    
    
    